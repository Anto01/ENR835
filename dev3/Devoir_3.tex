
% Default to the notebook output style

    


% Inherit from the specified cell style.




    
\documentclass{article}

    
    
    \usepackage{graphicx} % Used to insert images
    \usepackage{adjustbox} % Used to constrain images to a maximum size 
    \usepackage{color} % Allow colors to be defined
    \usepackage{enumerate} % Needed for markdown enumerations to work
    \usepackage{geometry} % Used to adjust the document margins
    \usepackage{amsmath} % Equations
    \usepackage{amssymb} % Equations
    \usepackage{eurosym} % defines \euro
    \usepackage[mathletters]{ucs} % Extended unicode (utf-8) support
    \usepackage[utf8x]{inputenc} % Allow utf-8 characters in the tex document
    \usepackage{fancyvrb} % verbatim replacement that allows latex
    \usepackage{grffile} % extends the file name processing of package graphics 
                         % to support a larger range 
    % The hyperref package gives us a pdf with properly built
    % internal navigation ('pdf bookmarks' for the table of contents,
    % internal cross-reference links, web links for URLs, etc.)
    \usepackage{hyperref}
    \usepackage{longtable} % longtable support required by pandoc >1.10
    \usepackage{booktabs}  % table support for pandoc > 1.12.2
    

    
    
    \definecolor{orange}{cmyk}{0,0.4,0.8,0.2}
    \definecolor{darkorange}{rgb}{.71,0.21,0.01}
    \definecolor{darkgreen}{rgb}{.12,.54,.11}
    \definecolor{myteal}{rgb}{.26, .44, .56}
    \definecolor{gray}{gray}{0.45}
    \definecolor{lightgray}{gray}{.95}
    \definecolor{mediumgray}{gray}{.8}
    \definecolor{inputbackground}{rgb}{.95, .95, .85}
    \definecolor{outputbackground}{rgb}{.95, .95, .95}
    \definecolor{traceback}{rgb}{1, .95, .95}
    % ansi colors
    \definecolor{red}{rgb}{.6,0,0}
    \definecolor{green}{rgb}{0,.65,0}
    \definecolor{brown}{rgb}{0.6,0.6,0}
    \definecolor{blue}{rgb}{0,.145,.698}
    \definecolor{purple}{rgb}{.698,.145,.698}
    \definecolor{cyan}{rgb}{0,.698,.698}
    \definecolor{lightgray}{gray}{0.5}
    
    % bright ansi colors
    \definecolor{darkgray}{gray}{0.25}
    \definecolor{lightred}{rgb}{1.0,0.39,0.28}
    \definecolor{lightgreen}{rgb}{0.48,0.99,0.0}
    \definecolor{lightblue}{rgb}{0.53,0.81,0.92}
    \definecolor{lightpurple}{rgb}{0.87,0.63,0.87}
    \definecolor{lightcyan}{rgb}{0.5,1.0,0.83}
    
    % commands and environments needed by pandoc snippets
    % extracted from the output of `pandoc -s`
    \providecommand{\tightlist}{%
      \setlength{\itemsep}{0pt}\setlength{\parskip}{0pt}}
    \DefineVerbatimEnvironment{Highlighting}{Verbatim}{commandchars=\\\{\}}
    % Add ',fontsize=\small' for more characters per line
    \newenvironment{Shaded}{}{}
    \newcommand{\KeywordTok}[1]{\textcolor[rgb]{0.00,0.44,0.13}{\textbf{{#1}}}}
    \newcommand{\DataTypeTok}[1]{\textcolor[rgb]{0.56,0.13,0.00}{{#1}}}
    \newcommand{\DecValTok}[1]{\textcolor[rgb]{0.25,0.63,0.44}{{#1}}}
    \newcommand{\BaseNTok}[1]{\textcolor[rgb]{0.25,0.63,0.44}{{#1}}}
    \newcommand{\FloatTok}[1]{\textcolor[rgb]{0.25,0.63,0.44}{{#1}}}
    \newcommand{\CharTok}[1]{\textcolor[rgb]{0.25,0.44,0.63}{{#1}}}
    \newcommand{\StringTok}[1]{\textcolor[rgb]{0.25,0.44,0.63}{{#1}}}
    \newcommand{\CommentTok}[1]{\textcolor[rgb]{0.38,0.63,0.69}{\textit{{#1}}}}
    \newcommand{\OtherTok}[1]{\textcolor[rgb]{0.00,0.44,0.13}{{#1}}}
    \newcommand{\AlertTok}[1]{\textcolor[rgb]{1.00,0.00,0.00}{\textbf{{#1}}}}
    \newcommand{\FunctionTok}[1]{\textcolor[rgb]{0.02,0.16,0.49}{{#1}}}
    \newcommand{\RegionMarkerTok}[1]{{#1}}
    \newcommand{\ErrorTok}[1]{\textcolor[rgb]{1.00,0.00,0.00}{\textbf{{#1}}}}
    \newcommand{\NormalTok}[1]{{#1}}
    
    % Define a nice break command that doesn't care if a line doesn't already
    % exist.
    \def\br{\hspace*{\fill} \\* }
    % Math Jax compatability definitions
    \def\gt{>}
    \def\lt{<}
    % Document parameters
    \title{Devoir\_3}
    
    
    

    % Pygments definitions
    
\makeatletter
\def\PY@reset{\let\PY@it=\relax \let\PY@bf=\relax%
    \let\PY@ul=\relax \let\PY@tc=\relax%
    \let\PY@bc=\relax \let\PY@ff=\relax}
\def\PY@tok#1{\csname PY@tok@#1\endcsname}
\def\PY@toks#1+{\ifx\relax#1\empty\else%
    \PY@tok{#1}\expandafter\PY@toks\fi}
\def\PY@do#1{\PY@bc{\PY@tc{\PY@ul{%
    \PY@it{\PY@bf{\PY@ff{#1}}}}}}}
\def\PY#1#2{\PY@reset\PY@toks#1+\relax+\PY@do{#2}}

\expandafter\def\csname PY@tok@mi\endcsname{\def\PY@tc##1{\textcolor[rgb]{0.40,0.40,0.40}{##1}}}
\expandafter\def\csname PY@tok@gs\endcsname{\let\PY@bf=\textbf}
\expandafter\def\csname PY@tok@sx\endcsname{\def\PY@tc##1{\textcolor[rgb]{0.00,0.50,0.00}{##1}}}
\expandafter\def\csname PY@tok@no\endcsname{\def\PY@tc##1{\textcolor[rgb]{0.53,0.00,0.00}{##1}}}
\expandafter\def\csname PY@tok@sh\endcsname{\def\PY@tc##1{\textcolor[rgb]{0.73,0.13,0.13}{##1}}}
\expandafter\def\csname PY@tok@go\endcsname{\def\PY@tc##1{\textcolor[rgb]{0.53,0.53,0.53}{##1}}}
\expandafter\def\csname PY@tok@nc\endcsname{\let\PY@bf=\textbf\def\PY@tc##1{\textcolor[rgb]{0.00,0.00,1.00}{##1}}}
\expandafter\def\csname PY@tok@nd\endcsname{\def\PY@tc##1{\textcolor[rgb]{0.67,0.13,1.00}{##1}}}
\expandafter\def\csname PY@tok@o\endcsname{\def\PY@tc##1{\textcolor[rgb]{0.40,0.40,0.40}{##1}}}
\expandafter\def\csname PY@tok@nv\endcsname{\def\PY@tc##1{\textcolor[rgb]{0.10,0.09,0.49}{##1}}}
\expandafter\def\csname PY@tok@c\endcsname{\let\PY@it=\textit\def\PY@tc##1{\textcolor[rgb]{0.25,0.50,0.50}{##1}}}
\expandafter\def\csname PY@tok@nn\endcsname{\let\PY@bf=\textbf\def\PY@tc##1{\textcolor[rgb]{0.00,0.00,1.00}{##1}}}
\expandafter\def\csname PY@tok@gt\endcsname{\def\PY@tc##1{\textcolor[rgb]{0.00,0.27,0.87}{##1}}}
\expandafter\def\csname PY@tok@na\endcsname{\def\PY@tc##1{\textcolor[rgb]{0.49,0.56,0.16}{##1}}}
\expandafter\def\csname PY@tok@gr\endcsname{\def\PY@tc##1{\textcolor[rgb]{1.00,0.00,0.00}{##1}}}
\expandafter\def\csname PY@tok@c1\endcsname{\let\PY@it=\textit\def\PY@tc##1{\textcolor[rgb]{0.25,0.50,0.50}{##1}}}
\expandafter\def\csname PY@tok@kc\endcsname{\let\PY@bf=\textbf\def\PY@tc##1{\textcolor[rgb]{0.00,0.50,0.00}{##1}}}
\expandafter\def\csname PY@tok@si\endcsname{\let\PY@bf=\textbf\def\PY@tc##1{\textcolor[rgb]{0.73,0.40,0.53}{##1}}}
\expandafter\def\csname PY@tok@vi\endcsname{\def\PY@tc##1{\textcolor[rgb]{0.10,0.09,0.49}{##1}}}
\expandafter\def\csname PY@tok@kd\endcsname{\let\PY@bf=\textbf\def\PY@tc##1{\textcolor[rgb]{0.00,0.50,0.00}{##1}}}
\expandafter\def\csname PY@tok@s2\endcsname{\def\PY@tc##1{\textcolor[rgb]{0.73,0.13,0.13}{##1}}}
\expandafter\def\csname PY@tok@vg\endcsname{\def\PY@tc##1{\textcolor[rgb]{0.10,0.09,0.49}{##1}}}
\expandafter\def\csname PY@tok@nb\endcsname{\def\PY@tc##1{\textcolor[rgb]{0.00,0.50,0.00}{##1}}}
\expandafter\def\csname PY@tok@sr\endcsname{\def\PY@tc##1{\textcolor[rgb]{0.73,0.40,0.53}{##1}}}
\expandafter\def\csname PY@tok@s1\endcsname{\def\PY@tc##1{\textcolor[rgb]{0.73,0.13,0.13}{##1}}}
\expandafter\def\csname PY@tok@sb\endcsname{\def\PY@tc##1{\textcolor[rgb]{0.73,0.13,0.13}{##1}}}
\expandafter\def\csname PY@tok@ch\endcsname{\let\PY@it=\textit\def\PY@tc##1{\textcolor[rgb]{0.25,0.50,0.50}{##1}}}
\expandafter\def\csname PY@tok@sc\endcsname{\def\PY@tc##1{\textcolor[rgb]{0.73,0.13,0.13}{##1}}}
\expandafter\def\csname PY@tok@kn\endcsname{\let\PY@bf=\textbf\def\PY@tc##1{\textcolor[rgb]{0.00,0.50,0.00}{##1}}}
\expandafter\def\csname PY@tok@vc\endcsname{\def\PY@tc##1{\textcolor[rgb]{0.10,0.09,0.49}{##1}}}
\expandafter\def\csname PY@tok@cs\endcsname{\let\PY@it=\textit\def\PY@tc##1{\textcolor[rgb]{0.25,0.50,0.50}{##1}}}
\expandafter\def\csname PY@tok@ne\endcsname{\let\PY@bf=\textbf\def\PY@tc##1{\textcolor[rgb]{0.82,0.25,0.23}{##1}}}
\expandafter\def\csname PY@tok@gi\endcsname{\def\PY@tc##1{\textcolor[rgb]{0.00,0.63,0.00}{##1}}}
\expandafter\def\csname PY@tok@s\endcsname{\def\PY@tc##1{\textcolor[rgb]{0.73,0.13,0.13}{##1}}}
\expandafter\def\csname PY@tok@ge\endcsname{\let\PY@it=\textit}
\expandafter\def\csname PY@tok@cpf\endcsname{\let\PY@it=\textit\def\PY@tc##1{\textcolor[rgb]{0.25,0.50,0.50}{##1}}}
\expandafter\def\csname PY@tok@ow\endcsname{\let\PY@bf=\textbf\def\PY@tc##1{\textcolor[rgb]{0.67,0.13,1.00}{##1}}}
\expandafter\def\csname PY@tok@kr\endcsname{\let\PY@bf=\textbf\def\PY@tc##1{\textcolor[rgb]{0.00,0.50,0.00}{##1}}}
\expandafter\def\csname PY@tok@mh\endcsname{\def\PY@tc##1{\textcolor[rgb]{0.40,0.40,0.40}{##1}}}
\expandafter\def\csname PY@tok@nf\endcsname{\def\PY@tc##1{\textcolor[rgb]{0.00,0.00,1.00}{##1}}}
\expandafter\def\csname PY@tok@gp\endcsname{\let\PY@bf=\textbf\def\PY@tc##1{\textcolor[rgb]{0.00,0.00,0.50}{##1}}}
\expandafter\def\csname PY@tok@cp\endcsname{\def\PY@tc##1{\textcolor[rgb]{0.74,0.48,0.00}{##1}}}
\expandafter\def\csname PY@tok@k\endcsname{\let\PY@bf=\textbf\def\PY@tc##1{\textcolor[rgb]{0.00,0.50,0.00}{##1}}}
\expandafter\def\csname PY@tok@il\endcsname{\def\PY@tc##1{\textcolor[rgb]{0.40,0.40,0.40}{##1}}}
\expandafter\def\csname PY@tok@err\endcsname{\def\PY@bc##1{\setlength{\fboxsep}{0pt}\fcolorbox[rgb]{1.00,0.00,0.00}{1,1,1}{\strut ##1}}}
\expandafter\def\csname PY@tok@nl\endcsname{\def\PY@tc##1{\textcolor[rgb]{0.63,0.63,0.00}{##1}}}
\expandafter\def\csname PY@tok@bp\endcsname{\def\PY@tc##1{\textcolor[rgb]{0.00,0.50,0.00}{##1}}}
\expandafter\def\csname PY@tok@mb\endcsname{\def\PY@tc##1{\textcolor[rgb]{0.40,0.40,0.40}{##1}}}
\expandafter\def\csname PY@tok@gh\endcsname{\let\PY@bf=\textbf\def\PY@tc##1{\textcolor[rgb]{0.00,0.00,0.50}{##1}}}
\expandafter\def\csname PY@tok@gd\endcsname{\def\PY@tc##1{\textcolor[rgb]{0.63,0.00,0.00}{##1}}}
\expandafter\def\csname PY@tok@ni\endcsname{\let\PY@bf=\textbf\def\PY@tc##1{\textcolor[rgb]{0.60,0.60,0.60}{##1}}}
\expandafter\def\csname PY@tok@cm\endcsname{\let\PY@it=\textit\def\PY@tc##1{\textcolor[rgb]{0.25,0.50,0.50}{##1}}}
\expandafter\def\csname PY@tok@m\endcsname{\def\PY@tc##1{\textcolor[rgb]{0.40,0.40,0.40}{##1}}}
\expandafter\def\csname PY@tok@nt\endcsname{\let\PY@bf=\textbf\def\PY@tc##1{\textcolor[rgb]{0.00,0.50,0.00}{##1}}}
\expandafter\def\csname PY@tok@gu\endcsname{\let\PY@bf=\textbf\def\PY@tc##1{\textcolor[rgb]{0.50,0.00,0.50}{##1}}}
\expandafter\def\csname PY@tok@sd\endcsname{\let\PY@it=\textit\def\PY@tc##1{\textcolor[rgb]{0.73,0.13,0.13}{##1}}}
\expandafter\def\csname PY@tok@ss\endcsname{\def\PY@tc##1{\textcolor[rgb]{0.10,0.09,0.49}{##1}}}
\expandafter\def\csname PY@tok@mo\endcsname{\def\PY@tc##1{\textcolor[rgb]{0.40,0.40,0.40}{##1}}}
\expandafter\def\csname PY@tok@kp\endcsname{\def\PY@tc##1{\textcolor[rgb]{0.00,0.50,0.00}{##1}}}
\expandafter\def\csname PY@tok@se\endcsname{\let\PY@bf=\textbf\def\PY@tc##1{\textcolor[rgb]{0.73,0.40,0.13}{##1}}}
\expandafter\def\csname PY@tok@mf\endcsname{\def\PY@tc##1{\textcolor[rgb]{0.40,0.40,0.40}{##1}}}
\expandafter\def\csname PY@tok@kt\endcsname{\def\PY@tc##1{\textcolor[rgb]{0.69,0.00,0.25}{##1}}}
\expandafter\def\csname PY@tok@w\endcsname{\def\PY@tc##1{\textcolor[rgb]{0.73,0.73,0.73}{##1}}}

\def\PYZbs{\char`\\}
\def\PYZus{\char`\_}
\def\PYZob{\char`\{}
\def\PYZcb{\char`\}}
\def\PYZca{\char`\^}
\def\PYZam{\char`\&}
\def\PYZlt{\char`\<}
\def\PYZgt{\char`\>}
\def\PYZsh{\char`\#}
\def\PYZpc{\char`\%}
\def\PYZdl{\char`\$}
\def\PYZhy{\char`\-}
\def\PYZsq{\char`\'}
\def\PYZdq{\char`\"}
\def\PYZti{\char`\~}
% for compatibility with earlier versions
\def\PYZat{@}
\def\PYZlb{[}
\def\PYZrb{]}
\makeatother


    % Exact colors from NB
    \definecolor{incolor}{rgb}{0.0, 0.0, 0.5}
    \definecolor{outcolor}{rgb}{0.545, 0.0, 0.0}



    
    % Prevent overflowing lines due to hard-to-break entities
    \sloppy 
    % Setup hyperref package
    \hypersetup{
      breaklinks=true,  % so long urls are correctly broken across lines
      colorlinks=true,
      urlcolor=blue,
      linkcolor=darkorange,
      citecolor=darkgreen,
      }
    % Slightly bigger margins than the latex defaults
    
    \geometry{verbose,tmargin=1in,bmargin=1in,lmargin=1in,rmargin=1in}
    
    

    \begin{document}
    
    
    \maketitle
    
    

    
    \section{Devoir \#3 ENR382}\label{devoir-3-enr382}

    \subsection{Modules à importer}\label{modules-uxe0-importer}

    \begin{Verbatim}[commandchars=\\\{\}]
{\color{incolor}In [{\color{incolor}216}]:} \PY{c+c1}{\PYZsh{} Python}
          \PY{k+kn}{import} \PY{n+nn}{numpy} \PY{k}{as} \PY{n+nn}{np}
          \PY{k+kn}{import} \PY{n+nn}{scipy} \PY{k}{as} \PY{n+nn}{sp}
          \PY{k+kn}{import} \PY{n+nn}{scipy}\PY{n+nn}{.}\PY{n+nn}{constants} \PY{k}{as} \PY{n+nn}{const}
          \PY{o}{\PYZpc{}}\PY{k}{matplotlib} inline
          \PY{k+kn}{import} \PY{n+nn}{matplotlib}\PY{n+nn}{.}\PY{n+nn}{pyplot} \PY{k}{as} \PY{n+nn}{plt}
          \PY{k+kn}{import} \PY{n+nn}{pandas} \PY{k}{as} \PY{n+nn}{pd}
          \PY{k+kn}{import} \PY{n+nn}{pylab}
          \PY{k+kn}{import} \PY{n+nn}{sys}
          
          \PY{k+kn}{import} \PY{n+nn}{plotly}
          \PY{k+kn}{from} \PY{n+nn}{plotly}\PY{n+nn}{.}\PY{n+nn}{offline} \PY{k}{import} \PY{n}{plot}
          \PY{n}{init\PYZus{}notebook\PYZus{}mode}\PY{p}{(}\PY{p}{)}
          
          \PY{c+c1}{\PYZsh{} Externe}
          \PY{k+kn}{import} \PY{n+nn}{solar\PYZus{}mod} \PY{k}{as} \PY{n+nn}{sm}
          \PY{k+kn}{import} \PY{n+nn}{properties\PYZus{}mod} \PY{k}{as} \PY{n+nn}{pm}
\end{Verbatim}

    
    \begin{verbatim}
<IPython.core.display.HTML object>
    \end{verbatim}

    
    \subsection{Question 1}\label{question-1}

\subsubsection{Importation et apperçu des données du devoir
\#2}\label{importation-et-apperuxe7u-des-donnuxe9es-du-devoir-2}

    \begin{Verbatim}[commandchars=\\\{\}]
{\color{incolor}In [{\color{incolor}217}]:} \PY{n}{data\PYZus{}d2} \PY{o}{=} \PY{n}{pd}\PY{o}{.}\PY{n}{read\PYZus{}csv}\PY{p}{(}\PY{l+s+s1}{\PYZsq{}}\PY{l+s+s1}{../dev2/data\PYZus{}devoir2.csv}\PY{l+s+s1}{\PYZsq{}}\PY{p}{)}
          \PY{n+nb}{print}\PY{p}{(}\PY{n}{data\PYZus{}d2}\PY{p}{)}
          \PY{c+c1}{\PYZsh{}data\PYZus{}d2.head()}
\end{Verbatim}

    \begin{Verbatim}[commandchars=\\\{\}]
I  Ibn        Rb        omen        thez
0     0    0  2.728682 -185.932358  111.134497
1     0    0  2.746104 -170.932358  110.852047
2     0    0  2.958693 -155.932358  107.937769
3     0    0  3.590436 -140.932358  102.703818
4     0    0  6.351798 -125.932358   95.609050
5    27   98  0.000000 -110.932358   87.122329
6    99  110  0.000000  -95.932358   77.652628
7   280  411  0.109534  -80.932358   67.541652
8   432  499  0.487013  -65.932358   57.093909
9   543  423  0.673706  -50.932358   46.638033
10  636  387  0.775716  -35.932358   36.652838
11  802  579  0.830688  -20.932358   28.070626
12  796  521  0.854040   -5.932358   22.861140
13  791  519  0.851393    9.067642   23.530648
14  791  578  0.822137   24.067642   29.671890
15  609  348  0.759068   39.067642   38.665950
16  396  142  0.643653   54.067642   48.804472
17  339  175  0.429968   69.067642   59.290404
18  226  285  0.000000   84.067642   69.692861
19   83   86  0.000000   99.067642   79.695532
20   16    0  0.000000  114.067642   88.989998
21    0    0  5.245744  129.067642   97.222014
22    0    0  3.393977  144.067642  103.967806
23    0    0  2.892482  159.067642  108.752307
    \end{Verbatim}

    \subsubsection{Données du problème}\label{donnuxe9es-du-probluxe8me}

    \begin{Verbatim}[commandchars=\\\{\}]
{\color{incolor}In [{\color{incolor}218}]:} \PY{n}{T\PYZus{}pc} \PY{o}{=} \PY{l+m+mf}{52.0}         \PY{c+c1}{\PYZsh{} Temperature de la plaque (Celsius)}
          \PY{n}{T\PYZus{}pk} \PY{o}{=} \PY{n}{T\PYZus{}pc}\PY{o}{+}\PY{l+m+mi}{273}     \PY{c+c1}{\PYZsh{} Temperature de la plaque (Kelvin)}
          \PY{n}{T\PYZus{}ac} \PY{o}{=} \PY{l+m+mi}{18}           \PY{c+c1}{\PYZsh{} Temperature ambiante (celsius)}
          \PY{n}{T\PYZus{}ak} \PY{o}{=} \PY{n}{T\PYZus{}ac}\PY{o}{+}\PY{l+m+mi}{273}     \PY{c+c1}{\PYZsh{} Temperature ambiante (kelvin)}
          \PY{n}{h\PYZus{}w} \PY{o}{=} \PY{l+m+mf}{6.0}             \PY{c+c1}{\PYZsh{} coefficient de convection extérieur}
          \PY{n}{Lair}  \PY{o}{=} \PY{l+m+mf}{0.030}       \PY{c+c1}{\PYZsh{} epaisseur d\PYZsq{}air}
          \PY{n}{alpha\PYZus{}n} \PY{o}{=} \PY{l+m+mf}{0.95}      \PY{c+c1}{\PYZsh{} Coef d\PYZsq{}absortion solaire}
          \PY{n}{g} \PY{o}{=} \PY{l+m+mf}{9.8}  
          \PY{n}{rhog} \PY{o}{=} \PY{l+m+mf}{0.4}
          \PY{n}{KL} \PY{o}{=} \PY{l+m+mf}{0.0125}         \PY{c+c1}{\PYZsh{} Propriété du vitrage}
          \PY{n}{n2}\PY{o}{=}\PY{l+m+mf}{1.526}            \PY{c+c1}{\PYZsh{} indice de réfraction de la vitre}
          \PY{n}{n1}\PY{o}{=}\PY{l+m+mi}{1}                \PY{c+c1}{\PYZsh{} indice de réfraction de l\PYZsq{}air}
          \PY{n}{eps\PYZus{}p} \PY{o}{=} \PY{l+m+mf}{0.17}        \PY{c+c1}{\PYZsh{} Emissivité de la plaque}
          \PY{n}{eps\PYZus{}c} \PY{o}{=} \PY{l+m+mf}{0.88}         \PY{c+c1}{\PYZsh{} Emissivitée du verre}
          \PY{n}{N} \PY{o}{=} \PY{l+m+mi}{1}               \PY{c+c1}{\PYZsh{} Nombre de vitrage}
          \PY{n}{beta} \PY{o}{=} \PY{l+m+mi}{60}         \PY{c+c1}{\PYZsh{} Angle à plat}
          \PY{n}{gam} \PY{o}{=} \PY{l+m+mi}{0}             \PY{c+c1}{\PYZsh{} plein sud}
          \PY{n}{phi}  \PY{o}{=} \PY{l+m+mf}{45.0} \PY{o}{+}\PY{l+m+mf}{30.0}\PY{o}{/}\PY{l+m+mf}{60.0}  \PY{c+c1}{\PYZsh{} latitude  Montreal (45 deg 30 min nord)}
\end{Verbatim}

    \subsubsection{Sélection des données de la 12e
tranche}\label{suxe9lection-des-donnuxe9es-de-la-12e-tranche}

    \begin{Verbatim}[commandchars=\\\{\}]
{\color{incolor}In [{\color{incolor}219}]:} \PY{n}{tranche} \PY{o}{=} \PY{l+m+mi}{11}
          \PY{n}{I} \PY{o}{=} \PY{n}{data\PYZus{}d2}\PY{o}{.}\PY{n}{I}\PY{p}{[}\PY{n}{tranche}\PY{p}{]}
          \PY{n}{Ibn} \PY{o}{=} \PY{n}{data\PYZus{}d2}\PY{o}{.}\PY{n}{Ibn}\PY{p}{[}\PY{n}{tranche}\PY{p}{]}
          \PY{n}{n} \PY{o}{=}\PY{n}{sm}\PY{o}{.}\PY{n}{jour\PYZus{}mois\PYZus{}jour\PYZus{}annee}\PY{p}{(}\PY{l+m+mi}{12}\PY{p}{,}\PY{l+s+s1}{\PYZsq{}}\PY{l+s+s1}{juin}\PY{l+s+s1}{\PYZsq{}}\PY{p}{)}
          \PY{n}{thez} \PY{o}{=} \PY{n}{data\PYZus{}d2}\PY{o}{.}\PY{n}{thez}\PY{p}{[}\PY{n}{tranche}\PY{p}{]}
          \PY{n}{delt}  \PY{o}{=} \PY{n}{sm}\PY{o}{.}\PY{n}{decl\PYZus{}solaire}\PY{p}{(}\PY{n}{n}\PY{p}{)}
          \PY{n}{omen} \PY{o}{=} \PY{n}{data\PYZus{}d2}\PY{o}{.}\PY{n}{omen}\PY{p}{[}\PY{n}{tranche}\PY{p}{]}
          \PY{n}{Ib} \PY{o}{=} \PY{n}{Ibn} \PY{o}{*} \PY{n}{sm}\PY{o}{.}\PY{n}{cosd}\PY{p}{(}\PY{n}{thez}\PY{p}{)}
          \PY{n}{Id} \PY{o}{=} \PY{n}{I}\PY{o}{\PYZhy{}}\PY{n}{Ib}
          \PY{n}{Rb} \PY{o}{=} \PY{n}{sm}\PY{o}{.}\PY{n}{calcul\PYZus{}Rb}\PY{p}{(}\PY{n}{phi}\PY{p}{,}\PY{n}{n}\PY{p}{,}\PY{n}{omen}\PY{p}{,}\PY{n}{beta}\PY{p}{,}\PY{n}{gam}\PY{p}{)}
          
          \PY{n}{It}\PY{p}{,}\PY{n}{Ib}\PY{p}{,}\PY{n}{Id}\PY{p}{,}\PY{n}{Ir} \PY{o}{=} \PY{n}{sm}\PY{o}{.}\PY{n}{modele\PYZus{}isotropique}\PY{p}{(}\PY{n}{I}\PY{p}{,}\PY{n}{Ib}\PY{p}{,}\PY{n}{Id}\PY{p}{,}\PY{n}{beta}\PY{p}{,}\PY{n}{Rb}\PY{p}{,}\PY{n}{rhog}\PY{p}{)}
          
          \PY{n+nb}{print}\PY{p}{(}\PY{l+s+s1}{\PYZsq{}}\PY{l+s+s1}{omen =}\PY{l+s+s1}{\PYZsq{}}\PY{p}{,}\PY{n}{omen}\PY{p}{)}
          \PY{n+nb}{print}\PY{p}{(}\PY{l+s+s1}{\PYZsq{}}\PY{l+s+s1}{Rb = }\PY{l+s+s1}{\PYZsq{}}\PY{p}{,}\PY{n}{Rb}\PY{p}{)}
          \PY{n+nb}{print}\PY{p}{(}\PY{l+s+s1}{\PYZsq{}}\PY{l+s+s1}{It = }\PY{l+s+s1}{\PYZsq{}}\PY{p}{,}\PY{n}{It}\PY{p}{)}
          \PY{n+nb}{print}\PY{p}{(}\PY{l+s+s1}{\PYZsq{}}\PY{l+s+s1}{Ib = }\PY{l+s+s1}{\PYZsq{}}\PY{p}{,}\PY{n}{Ib}\PY{p}{)}
          \PY{n+nb}{print}\PY{p}{(}\PY{l+s+s1}{\PYZsq{}}\PY{l+s+s1}{Id = }\PY{l+s+s1}{\PYZsq{}}\PY{p}{,}\PY{n}{Id}\PY{p}{)}
          \PY{n+nb}{print}\PY{p}{(}\PY{l+s+s1}{\PYZsq{}}\PY{l+s+s1}{Ir =}\PY{l+s+s1}{\PYZsq{}}\PY{p}{,}\PY{n}{Ir}\PY{p}{)}
          \PY{n+nb}{print}\PY{p}{(}\PY{l+s+s1}{\PYZsq{}}\PY{l+s+s1}{thez = }\PY{l+s+s1}{\PYZsq{}}\PY{p}{,}\PY{n}{thez}\PY{p}{)}
\end{Verbatim}

    \begin{Verbatim}[commandchars=\\\{\}]
omen = -20.932357638
Rb =  0.83068791593
It =  722.922746543
Ib =  424.391149777
Id =  218.331596766
Ir = 80.2
thez =  28.0706255912
    \end{Verbatim}

    \subsubsection{Calculs du problème}\label{calculs-du-probluxe8me}

\paragraph{Calcul des angles réfléchi et
diffus}\label{calcul-des-angles-ruxe9fluxe9chi-et-diffus}

    \begin{Verbatim}[commandchars=\\\{\}]
{\color{incolor}In [{\color{incolor}220}]:} \PY{n}{the\PYZus{}g} \PY{o}{=} \PY{n}{sm}\PY{o}{.}\PY{n}{angle\PYZus{}reflechi}\PY{p}{(}\PY{n}{beta}\PY{p}{)}
          \PY{n}{the\PYZus{}d} \PY{o}{=} \PY{n}{sm}\PY{o}{.}\PY{n}{angle\PYZus{}diffus}\PY{p}{(}\PY{n}{beta}\PY{p}{)}
          \PY{c+c1}{\PYZsh{}the = abs(thez\PYZhy{}beta)}
          
          \PY{n}{the} \PY{o}{=} \PY{n}{sm}\PY{o}{.}\PY{n}{normale\PYZus{}solaire}\PY{p}{(}\PY{n}{delt}\PY{p}{,}\PY{n}{phi}\PY{p}{,}\PY{n}{omen}\PY{p}{,}\PY{n}{beta}\PY{p}{,}\PY{n}{gam}\PY{p}{)}
          
          \PY{n+nb}{print}\PY{p}{(}\PY{l+s+s1}{\PYZsq{}}\PY{l+s+s1}{the = }\PY{l+s+s1}{\PYZsq{}}\PY{p}{,}\PY{n}{the}\PY{p}{)}
          \PY{n+nb}{print}\PY{p}{(}\PY{l+s+s1}{\PYZsq{}}\PY{l+s+s1}{the\PYZus{}g = }\PY{l+s+s1}{\PYZsq{}}\PY{p}{,}\PY{n}{the\PYZus{}g}\PY{p}{)}
          \PY{n+nb}{print}\PY{p}{(}\PY{l+s+s1}{\PYZsq{}}\PY{l+s+s1}{the\PYZus{}d = }\PY{l+s+s1}{\PYZsq{}}\PY{p}{,}\PY{n}{the\PYZus{}d}\PY{p}{)}
\end{Verbatim}

    \begin{Verbatim}[commandchars=\\\{\}]
the =  42.8638179755
the\_g =  64.9668
the\_d =  56.7612
    \end{Verbatim}

    

    

    \paragraph{Calcul du coéfficient absorbeur pour faible longueur
d'ondes}\label{calcul-du-couxe9fficient-absorbeur-pour-faible-longueur-dondes}

    \begin{Verbatim}[commandchars=\\\{\}]
{\color{incolor}In [{\color{incolor}221}]:} \PY{n}{tau\PYZus{}al\PYZus{}b} \PY{o}{=} \PY{n}{sm}\PY{o}{.}\PY{n}{Calcul\PYZus{}tau\PYZus{}al}\PY{p}{(}\PY{n}{the}\PY{p}{,}\PY{n}{alpha\PYZus{}n}\PY{p}{,}\PY{n}{KL}\PY{p}{,}\PY{n}{n2}\PY{p}{,}\PY{n}{n1}\PY{p}{,}\PY{n}{N}\PY{p}{)}
          \PY{n}{tau\PYZus{}al\PYZus{}g} \PY{o}{=} \PY{n}{sm}\PY{o}{.}\PY{n}{Calcul\PYZus{}tau\PYZus{}al}\PY{p}{(}\PY{n}{the\PYZus{}g}\PY{p}{,}\PY{n}{alpha\PYZus{}n}\PY{p}{,}\PY{n}{KL}\PY{p}{,}\PY{n}{n2}\PY{p}{,}\PY{n}{n1}\PY{p}{,}\PY{n}{N}\PY{p}{)}
          \PY{n}{tau\PYZus{}al\PYZus{}d} \PY{o}{=} \PY{n}{sm}\PY{o}{.}\PY{n}{Calcul\PYZus{}tau\PYZus{}al}\PY{p}{(}\PY{n}{the\PYZus{}d}\PY{p}{,}\PY{n}{alpha\PYZus{}n}\PY{p}{,}\PY{n}{KL}\PY{p}{,}\PY{n}{n2}\PY{p}{,}\PY{n}{n1}\PY{p}{,}\PY{n}{N}\PY{p}{)}
          
          \PY{n+nb}{print}\PY{p}{(}\PY{l+s+s1}{\PYZsq{}}\PY{l+s+s1}{tau\PYZus{}al\PYZus{}b =}\PY{l+s+s1}{\PYZsq{}}\PY{p}{,} \PY{n}{tau\PYZus{}al\PYZus{}b}\PY{p}{)}
          \PY{n+nb}{print}\PY{p}{(}\PY{l+s+s1}{\PYZsq{}}\PY{l+s+s1}{tau\PYZus{}al\PYZus{}g =}\PY{l+s+s1}{\PYZsq{}}\PY{p}{,} \PY{n}{tau\PYZus{}al\PYZus{}g}\PY{p}{)}
          \PY{n+nb}{print}\PY{p}{(}\PY{l+s+s1}{\PYZsq{}}\PY{l+s+s1}{tau\PYZus{}al\PYZus{}d =}\PY{l+s+s1}{\PYZsq{}}\PY{p}{,} \PY{n}{tau\PYZus{}al\PYZus{}d}\PY{p}{)}
\end{Verbatim}

    \begin{Verbatim}[commandchars=\\\{\}]
tau\_al\_b = 0.832787232782
tau\_al\_g = 0.688465506364
tau\_al\_d = 0.772853382109
    \end{Verbatim}

    \paragraph{Calcul de la radiation totale
transmise}\label{calcul-de-la-radiation-totale-transmise}

    \begin{Verbatim}[commandchars=\\\{\}]
{\color{incolor}In [{\color{incolor}222}]:} \PY{n}{Ibt} \PY{o}{=} \PY{n}{Ib}\PY{o}{*}\PY{n}{tau\PYZus{}al\PYZus{}b}
          \PY{n}{Idt} \PY{o}{=} \PY{n}{Id}\PY{o}{*}\PY{n}{tau\PYZus{}al\PYZus{}d}
          \PY{n}{Irt} \PY{o}{=} \PY{n}{Ir}\PY{o}{*}\PY{n}{tau\PYZus{}al\PYZus{}g}
          \PY{n}{S} \PY{o}{=}  \PY{n}{Ibt}\PY{o}{+}\PY{n}{Idt}\PY{o}{+}\PY{n}{Irt}
          
          \PY{n+nb}{print}\PY{p}{(}\PY{l+s+s1}{\PYZsq{}}\PY{l+s+s1}{Radiation directe transmise =}\PY{l+s+s1}{\PYZsq{}}\PY{p}{,}\PY{n}{Ibt}\PY{p}{)}
          \PY{n+nb}{print}\PY{p}{(}\PY{l+s+s1}{\PYZsq{}}\PY{l+s+s1}{Radiation diffuse transmise=}\PY{l+s+s1}{\PYZsq{}}\PY{p}{,}\PY{n}{Idt}\PY{p}{)}
          \PY{n+nb}{print}\PY{p}{(}\PY{l+s+s1}{\PYZsq{}}\PY{l+s+s1}{Radiation réfléchie transmise =}\PY{l+s+s1}{\PYZsq{}}\PY{p}{,}\PY{n}{Irt}\PY{p}{)}
          \PY{n+nb}{print}\PY{p}{(}\PY{l+s+s1}{\PYZsq{}}\PY{l+s+s1}{Radiation totale transmise (S) =}\PY{l+s+s1}{\PYZsq{}}\PY{p}{,}\PY{n}{S}\PY{p}{)}
\end{Verbatim}

    \begin{Verbatim}[commandchars=\\\{\}]
Radiation directe transmise = 353.42753124
Radiation diffuse transmise= 168.738312982
Radiation réfléchie transmise = 55.2149336104
Radiation totale transmise (S) = 577.380777832
    \end{Verbatim}

    

    \paragraph{Calcul des pertes par model
itératif}\label{calcul-des-pertes-par-model-ituxe9ratif}

Pour une plaque simple, le coéfficient de perte vers le haut est exprimé
grâce à cette formule. 

    Premièrement, il faut trouver chacun des termes du dénominateur. Les
formules suivantes seront utilisées. Pour le coéfficient de radiation de
la plaque interne jusqu'à la surface externe, Pour ce qui est de la
radiation de la surface externe vers l'air ambiante, La convection entre
les deux surfaces (\$ h\_\{c,p-c\} \$) sera déterminé grâce à la
fonction suivante, Pour enfin trouver le coéfficient de température de
la face externe du capteur. 

    \begin{Verbatim}[commandchars=\\\{\}]
{\color{incolor}In [{\color{incolor}238}]:} \PY{n}{T\PYZus{}cc} \PY{o}{=} \PY{l+m+mi}{40} \PY{c+c1}{\PYZsh{} Température initiale (hypothèse)}
          \PY{n}{T\PYZus{}old} \PY{o}{=} \PY{l+m+mi}{1}
          \PY{n}{converg} \PY{o}{=} \PY{p}{[}\PY{p}{]}
          \PY{n}{x\PYZus{}it} \PY{o}{=} \PY{l+m+mi}{0}
          \PY{n}{x\PYZus{}conv} \PY{o}{=} \PY{p}{[}\PY{p}{]}
          
          \PY{k}{while} \PY{p}{(}\PY{n+nb}{abs}\PY{p}{(}\PY{n}{T\PYZus{}old}\PY{o}{\PYZhy{}}\PY{n}{T\PYZus{}cc}\PY{p}{)} \PY{o}{\PYZgt{}} \PY{l+m+mf}{0.0001}\PY{p}{)}\PY{p}{:}
              
              \PY{n}{converg}\PY{o}{.}\PY{n}{append}\PY{p}{(}\PY{n}{T\PYZus{}cc}\PY{p}{)}
              \PY{n}{T\PYZus{}old} \PY{o}{=} \PY{n}{T\PYZus{}cc}
          
              \PY{n}{T\PYZus{}ck} \PY{o}{=} \PY{n}{T\PYZus{}cc} \PY{o}{+} \PY{l+m+mf}{273.0}
              \PY{n}{T\PYZus{}avg} \PY{o}{=} \PY{p}{(}\PY{n}{T\PYZus{}ck}\PY{o}{+}\PY{n}{T\PYZus{}pk}\PY{p}{)}\PY{o}{/}\PY{l+m+mi}{2}
          
              \PY{n}{hr\PYZus{}pc} \PY{o}{=} \PY{p}{(}\PY{n}{const}\PY{o}{.}\PY{n}{sigma} \PY{o}{*} \PY{p}{(}\PY{n}{T\PYZus{}pk}\PY{o}{*}\PY{o}{*}\PY{l+m+mi}{2}\PY{o}{+}\PY{n}{T\PYZus{}ck}\PY{o}{*}\PY{o}{*}\PY{l+m+mi}{2}\PY{p}{)} \PY{o}{*} \PY{p}{(}\PY{n}{T\PYZus{}pk}\PY{o}{+}\PY{n}{T\PYZus{}ck}\PY{p}{)}\PY{p}{)}\PY{o}{/}\PY{p}{(}\PY{p}{(}\PY{l+m+mi}{1}\PY{o}{/}\PY{n}{eps\PYZus{}p}\PY{p}{)}\PY{o}{+}\PY{p}{(}\PY{l+m+mi}{1}\PY{o}{/}\PY{n}{eps\PYZus{}c}\PY{p}{)}\PY{o}{\PYZhy{}}\PY{l+m+mi}{1}\PY{p}{)}
              
              \PY{n}{hr\PYZus{}ca} \PY{o}{=} \PY{n}{eps\PYZus{}c}\PY{o}{*}\PY{n}{const}\PY{o}{.}\PY{n}{sigma}\PY{o}{*}\PY{p}{(}\PY{n}{T\PYZus{}ck}\PY{o}{*}\PY{o}{*}\PY{l+m+mi}{2}\PY{o}{+}\PY{n}{T\PYZus{}ak}\PY{o}{*}\PY{o}{*}\PY{l+m+mi}{2}\PY{p}{)}\PY{o}{*}\PY{p}{(}\PY{n}{T\PYZus{}ck}\PY{o}{+}\PY{n}{T\PYZus{}ak}\PY{p}{)}
          
              \PY{n}{nu} \PY{o}{=} \PY{n}{pm}\PY{o}{.}\PY{n}{air\PYZus{}prop}\PY{p}{(}\PY{l+s+s1}{\PYZsq{}}\PY{l+s+s1}{nu}\PY{l+s+s1}{\PYZsq{}}\PY{p}{,}\PY{n}{T\PYZus{}avg}\PY{p}{)}
              \PY{n}{al} \PY{o}{=} \PY{n}{pm}\PY{o}{.}\PY{n}{air\PYZus{}prop}\PY{p}{(}\PY{l+s+s1}{\PYZsq{}}\PY{l+s+s1}{al}\PY{l+s+s1}{\PYZsq{}}\PY{p}{,}\PY{n}{T\PYZus{}avg}\PY{p}{)}
              \PY{n}{k} \PY{o}{=} \PY{n}{pm}\PY{o}{.}\PY{n}{air\PYZus{}prop}\PY{p}{(}\PY{l+s+s1}{\PYZsq{}}\PY{l+s+s1}{k}\PY{l+s+s1}{\PYZsq{}}\PY{p}{,}\PY{n}{T\PYZus{}avg}\PY{p}{)}
              \PY{n}{Ra} \PY{o}{=} \PY{n}{const}\PY{o}{.}\PY{n}{g}\PY{o}{*}\PY{p}{(}\PY{l+m+mi}{1}\PY{o}{/}\PY{n}{T\PYZus{}avg}\PY{p}{)}\PY{o}{*}\PY{n+nb}{abs}\PY{p}{(}\PY{n}{T\PYZus{}ck}\PY{o}{\PYZhy{}}\PY{n}{T\PYZus{}pk}\PY{p}{)}\PY{o}{*}\PY{n}{Lair}\PY{o}{*}\PY{o}{*}\PY{l+m+mi}{3}\PY{o}{/}\PY{p}{(}\PY{n}{nu}\PY{o}{*}\PY{n}{al}\PY{p}{)}
          
              \PY{n}{f1} \PY{o}{=} \PY{n+nb}{max}\PY{p}{(}\PY{l+m+mi}{0}\PY{p}{,}\PY{l+m+mf}{1.0}\PY{o}{\PYZhy{}}\PY{l+m+mf}{1708.0}\PY{o}{/}\PY{p}{(}\PY{n}{Ra}\PY{o}{*}\PY{n}{sm}\PY{o}{.}\PY{n}{cosd}\PY{p}{(}\PY{n}{beta}\PY{p}{)}\PY{p}{)}\PY{p}{)}
              \PY{n}{f2} \PY{o}{=} \PY{l+m+mf}{1.0}\PY{o}{\PYZhy{}}\PY{l+m+mi}{1708}\PY{o}{*}\PY{p}{(}\PY{n}{sm}\PY{o}{.}\PY{n}{sind}\PY{p}{(}\PY{l+m+mf}{1.8}\PY{o}{*}\PY{n}{beta}\PY{p}{)}\PY{p}{)}\PY{o}{*}\PY{o}{*}\PY{l+m+mf}{1.6}\PY{o}{/}\PY{p}{(}\PY{n}{Ra}\PY{o}{*}\PY{n}{sm}\PY{o}{.}\PY{n}{cosd}\PY{p}{(}\PY{n}{beta}\PY{p}{)}\PY{p}{)}
              \PY{n}{f3} \PY{o}{=} \PY{n+nb}{max}\PY{p}{(}\PY{l+m+mi}{0}\PY{p}{,}\PY{p}{(}\PY{n}{Ra}\PY{o}{*}\PY{n}{sm}\PY{o}{.}\PY{n}{cosd}\PY{p}{(}\PY{n}{beta}\PY{p}{)}\PY{o}{/}\PY{l+m+mi}{5830}\PY{p}{)}\PY{o}{*}\PY{o}{*}\PY{p}{(}\PY{l+m+mf}{1.0}\PY{o}{/}\PY{l+m+mf}{3.0}\PY{p}{)}\PY{o}{\PYZhy{}}\PY{l+m+mf}{1.0}\PY{p}{)}
              \PY{n}{Nu} \PY{o}{=} \PY{l+m+mf}{1.0} \PY{o}{+} \PY{l+m+mf}{1.44}\PY{o}{*}\PY{n}{f1}\PY{o}{*}\PY{n}{f2}\PY{o}{+}\PY{n}{f3} 
              \PY{n}{hc\PYZus{}pc} \PY{o}{=} \PY{n}{Nu} \PY{o}{*} \PY{n}{k}\PY{o}{/}\PY{n}{Lair}
          
              \PY{n}{T\PYZus{}p} \PY{o}{=} \PY{n}{T\PYZus{}avg}\PY{o}{\PYZhy{}}\PY{l+m+mf}{273.0}
              
              
              \PY{n}{Ut} \PY{o}{=} \PY{p}{(}\PY{p}{(}\PY{l+m+mi}{1}\PY{o}{/}\PY{p}{(}\PY{n}{hc\PYZus{}pc} \PY{o}{+} \PY{n}{hr\PYZus{}pc}\PY{p}{)}\PY{p}{)}\PY{o}{+}\PY{p}{(}\PY{l+m+mi}{1}\PY{o}{/}\PY{p}{(}\PY{n}{h\PYZus{}w}\PY{o}{+}\PY{n}{hr\PYZus{}ca}\PY{p}{)}\PY{p}{)}\PY{p}{)}\PY{o}{*}\PY{o}{*}\PY{o}{\PYZhy{}}\PY{l+m+mi}{1}
              
              \PY{n}{T\PYZus{}cc} \PY{o}{=} \PY{n}{T\PYZus{}p} \PY{o}{\PYZhy{}} \PY{p}{(}\PY{p}{(}\PY{n}{Ut}\PY{o}{*}\PY{p}{(}\PY{n}{T\PYZus{}p}\PY{o}{\PYZhy{}}\PY{n}{T\PYZus{}ac}\PY{p}{)}\PY{p}{)}\PY{o}{/}\PY{p}{(}\PY{n}{hc\PYZus{}pc}\PY{o}{+}\PY{n}{hr\PYZus{}pc}\PY{p}{)}\PY{p}{)}
              \PY{n}{x\PYZus{}conv}\PY{o}{.}\PY{n}{append}\PY{p}{(}\PY{n}{x\PYZus{}it}\PY{p}{)}
              \PY{n}{x\PYZus{}it} \PY{o}{=} \PY{n}{x\PYZus{}it} \PY{o}{+} \PY{l+m+mi}{1}
          
          \PY{n}{q} \PY{o}{=} \PY{n}{Ut}\PY{o}{*}\PY{p}{(}\PY{n}{T\PYZus{}pc}\PY{o}{\PYZhy{}}\PY{n}{T\PYZus{}ac}\PY{p}{)}
          
          \PY{c+c1}{\PYZsh{}plotly.offline.iplot(\PYZob{}}
          \PY{c+c1}{\PYZsh{}\PYZdq{}data\PYZdq{}: [\PYZob{}}
          \PY{c+c1}{\PYZsh{}    \PYZdq{}x\PYZdq{}: x\PYZus{}conv,}
          \PY{c+c1}{\PYZsh{}    \PYZdq{}y\PYZdq{}: converg}
          \PY{c+c1}{\PYZsh{}\PYZcb{}],}
          \PY{c+c1}{\PYZsh{}\PYZdq{}layout\PYZdq{}: \PYZob{}}
          \PY{c+c1}{\PYZsh{}    \PYZdq{}title\PYZdq{}: \PYZdq{}Convergence de la température externe\PYZdq{}}
          \PY{c+c1}{\PYZsh{}\PYZcb{}}
          \PY{c+c1}{\PYZsh{}\PYZcb{})}
          
          
          
          \PY{n+nb}{print}\PY{p}{(}\PY{l+s+s1}{\PYZsq{}}\PY{l+s+s1}{Le coéfficient de perte vers le haut =}\PY{l+s+s1}{\PYZsq{}}\PY{p}{,}\PY{n}{Ut}\PY{p}{,} \PY{l+s+s1}{\PYZsq{}}\PY{l+s+s1}{W/m² °C}\PY{l+s+s1}{\PYZsq{}}\PY{p}{)}
          \PY{n+nb}{print}\PY{p}{(}\PY{l+s+s1}{\PYZsq{}}\PY{l+s+s1}{Les pertes vers le haut =}\PY{l+s+s1}{\PYZsq{}}\PY{p}{,} \PY{n}{q}\PY{p}{,} \PY{l+s+s1}{\PYZsq{}}\PY{l+s+s1}{W/m²}\PY{l+s+s1}{\PYZsq{}}\PY{p}{)}
          \PY{n+nb}{print}\PY{p}{(}\PY{l+s+s1}{\PYZsq{}}\PY{l+s+s1}{Ce qui est capté}\PY{l+s+s1}{\PYZsq{}}\PY{p}{,}\PY{n}{total\PYZus{}iteratif}\PY{p}{,}\PY{l+s+s1}{\PYZsq{}}\PY{l+s+s1}{W/m²}\PY{l+s+s1}{\PYZsq{}} \PY{p}{)}
          \PY{n+nb}{print}\PY{p}{(}\PY{l+s+s1}{\PYZsq{}}\PY{l+s+s1}{La température de la surface extérieure =}\PY{l+s+s1}{\PYZsq{}}\PY{p}{,}\PY{n}{T\PYZus{}cc}\PY{p}{,}\PY{l+s+s1}{\PYZsq{}}\PY{l+s+s1}{°C}\PY{l+s+s1}{\PYZsq{}}\PY{p}{)}
\end{Verbatim}

    \begin{Verbatim}[commandchars=\\\{\}]
Le coéfficient de perte vers le haut = 2.8585751874 W/m² °C
Les pertes vers le haut = 97.1915563716 W/m²
Ce qui est capté 480.189221461 W/m²
La température de la surface extérieure = 23.0521700256 °C
    \end{Verbatim}

    \paragraph{Calcul des pertes par modèle empirique
Klein}\label{calcul-des-pertes-par-moduxe8le-empirique-klein}

En plus de calculer avec la méthode itérative, la méthode Klein est
utilisé afin de comparer les résultats.

    

    \begin{Verbatim}[commandchars=\\\{\}]
{\color{incolor}In [{\color{incolor}237}]:} \PY{n}{pertes\PYZus{}empirique\PYZus{}coef} \PY{o}{=} \PY{n}{sm}\PY{o}{.}\PY{n}{U\PYZus{}Klein}\PY{p}{(}\PY{n}{T\PYZus{}pc}\PY{p}{,}\PY{n}{T\PYZus{}ac}\PY{p}{,}\PY{n}{beta}\PY{p}{,}\PY{n}{h\PYZus{}w}\PY{p}{,}\PY{n}{eps\PYZus{}p}\PY{p}{,}\PY{n}{eps\PYZus{}c}\PY{p}{,}\PY{n}{N}\PY{p}{)}
          \PY{n}{pertes\PYZus{}empirique} \PY{o}{=} \PY{n}{pertes\PYZus{}empirique\PYZus{}coef}\PY{o}{*}\PY{p}{(}\PY{n}{T\PYZus{}pc}\PY{o}{\PYZhy{}}\PY{n}{T\PYZus{}ac}\PY{p}{)}
          \PY{n}{total\PYZus{}empirique} \PY{o}{=} \PY{n}{S}\PY{o}{\PYZhy{}}\PY{n}{pertes\PYZus{}empirique}
          
          
          \PY{n+nb}{print}\PY{p}{(}\PY{l+s+s1}{\PYZsq{}}\PY{l+s+s1}{Coefficient de perte vers le haut Klein}\PY{l+s+s1}{\PYZsq{}}\PY{p}{,}\PY{n}{pertes\PYZus{}empirique\PYZus{}coef}\PY{p}{,}\PY{l+s+s1}{\PYZsq{}}\PY{l+s+s1}{W/m² °C}\PY{l+s+s1}{\PYZsq{}}\PY{p}{)}
          \PY{n+nb}{print}\PY{p}{(}\PY{l+s+s1}{\PYZsq{}}\PY{l+s+s1}{Pertes vers le haut =}\PY{l+s+s1}{\PYZsq{}}\PY{p}{,}\PY{n}{pertes\PYZus{}empirique}\PY{p}{,} \PY{l+s+s1}{\PYZsq{}}\PY{l+s+s1}{W/m²}\PY{l+s+s1}{\PYZsq{}}\PY{p}{)}
          \PY{n+nb}{print}\PY{p}{(}\PY{l+s+s1}{\PYZsq{}}\PY{l+s+s1}{Ce qui est capté =}\PY{l+s+s1}{\PYZsq{}}\PY{p}{,}\PY{n}{total\PYZus{}empirique}\PY{p}{,} \PY{l+s+s1}{\PYZsq{}}\PY{l+s+s1}{W/m²}\PY{l+s+s1}{\PYZsq{}}\PY{p}{)}
\end{Verbatim}

    \begin{Verbatim}[commandchars=\\\{\}]
Coefficient de perte vers le haut Klein 0.7025954760782759 W/m² °C
Pertes vers le haut = 23.88824618666138 W/m²
Ce qui est capté = 553.492531646 W/m²
    \end{Verbatim}

    \paragraph{Calcul du rendement du
capteur}\label{calcul-du-rendement-du-capteur}

    \begin{Verbatim}[commandchars=\\\{\}]
{\color{incolor}In [{\color{incolor}226}]:} \PY{n}{Rendement\PYZus{}empirique} \PY{o}{=} \PY{n}{total\PYZus{}empirique}\PY{o}{/}\PY{n}{It}
          \PY{n}{Rendement\PYZus{}iteratif} \PY{o}{=} \PY{n}{total\PYZus{}iteratif}\PY{o}{/}\PY{n}{It}
          \PY{n+nb}{print}\PY{p}{(}\PY{l+s+s1}{\PYZsq{}}\PY{l+s+s1}{Le rendement du capteur est de}\PY{l+s+s1}{\PYZsq{}}\PY{p}{,}\PY{n}{Rendement\PYZus{}empirique}\PY{p}{,}\PY{l+s+s1}{\PYZsq{}}\PY{l+s+s1}{pour la méthode empirique}\PY{l+s+s1}{\PYZsq{}}\PY{p}{)}
          \PY{n+nb}{print}\PY{p}{(}\PY{l+s+s1}{\PYZsq{}}\PY{l+s+s1}{Le rendement du capteur est de}\PY{l+s+s1}{\PYZsq{}}\PY{p}{,}\PY{n}{Rendement\PYZus{}iteratif}\PY{p}{,}\PY{l+s+s1}{\PYZsq{}}\PY{l+s+s1}{pour la méthode itérative}\PY{l+s+s1}{\PYZsq{}}\PY{p}{)}
\end{Verbatim}

    \begin{Verbatim}[commandchars=\\\{\}]
Le rendement du capteur est de 0.661970049777 pour la méthode empirique
Le rendement du capteur est de 0.664233106175 pour la méthode itérative
    \end{Verbatim}

    \subsubsection{Résultats}\label{ruxe9sultats}

    \begin{Verbatim}[commandchars=\\\{\}]
{\color{incolor}In [{\color{incolor}227}]:} \PY{n+nb}{print}\PY{p}{(}\PY{l+s+s1}{\PYZsq{}}\PY{l+s+s1}{a)}\PY{l+s+s1}{\PYZsq{}}\PY{p}{,}\PY{n}{S}\PY{p}{,}\PY{l+s+s1}{\PYZsq{}}\PY{l+s+s1}{J/m²}\PY{l+s+s1}{\PYZsq{}}\PY{p}{)}
          \PY{n+nb}{print}\PY{p}{(}\PY{l+s+s1}{\PYZsq{}}\PY{l+s+s1}{b)}\PY{l+s+s1}{\PYZsq{}}\PY{p}{,}\PY{n}{q}\PY{p}{,} \PY{l+s+s1}{\PYZsq{}}\PY{l+s+s1}{W/m²}\PY{l+s+s1}{\PYZsq{}}\PY{p}{)}
          \PY{n+nb}{print}\PY{p}{(}\PY{l+s+s1}{\PYZsq{}}\PY{l+s+s1}{c)}\PY{l+s+s1}{\PYZsq{}}\PY{p}{,}\PY{n}{Rendement\PYZus{}iteratif}\PY{p}{)}
\end{Verbatim}

    \begin{Verbatim}[commandchars=\\\{\}]
a) 577.380777832 J/m²
b) 97.1915563716 W/m²
c) 0.664233106175
    \end{Verbatim}

    \subsection{Question 2}\label{question-2}

    \subsubsection{Données du problème}\label{donnuxe9es-du-probluxe8me}

    \begin{Verbatim}[commandchars=\\\{\}]
{\color{incolor}In [{\color{incolor}228}]:} \PY{n}{H}\PY{o}{=} \PY{l+m+mf}{0.8}
          \PY{n}{N}\PY{o}{=}\PY{l+m+mi}{8}
          \PY{n}{Y}\PY{o}{=}\PY{l+m+mi}{2}
          \PY{n}{Ac}\PY{o}{=}\PY{n}{H}\PY{o}{*}\PY{n}{Y} \PY{c+c1}{\PYZsh{} Surface des capteurs}
          \PY{n}{W}\PY{o}{=}\PY{l+m+mf}{0.1} \PY{c+c1}{\PYZsh{} Distance entre les tubes (m)}
          \PY{n}{D} \PY{o}{=} \PY{l+m+mf}{0.007} \PY{c+c1}{\PYZsh{} m}
          \PY{n}{Cb}\PY{o}{=}\PY{l+m+mf}{100.0} \PY{c+c1}{\PYZsh{} W/mK}
          \PY{n}{P}\PY{o}{=}\PY{l+m+mi}{2}\PY{o}{*}\PY{p}{(}\PY{n}{H}\PY{o}{+}\PY{n}{Y}\PY{p}{)} \PY{c+c1}{\PYZsh{} Perimetre du capteur}
          \PY{n}{deltaa}\PY{o}{=} \PY{l+m+mf}{0.001} \PY{c+c1}{\PYZsh{}epaisseur de la plaque (m)}
          \PY{n}{ka}\PY{o}{=}\PY{l+m+mf}{400.0}  \PY{c+c1}{\PYZsh{} W/m2K}
          \PY{n}{mpt}\PY{o}{=}\PY{l+m+mf}{0.016} \PY{c+c1}{\PYZsh{} Debit total Kg/s}
          \PY{n}{hf}\PY{o}{=}\PY{l+m+mf}{1100.0} \PY{c+c1}{\PYZsh{}W/m2K}
          \PY{n}{Cp}\PY{o}{=}\PY{l+m+mf}{4180.0}   \PY{c+c1}{\PYZsh{}J/KgK}
          \PY{n}{Rpjoint} \PY{o}{=} \PY{l+m+mi}{1}\PY{o}{/}\PY{n}{Cb}    \PY{c+c1}{\PYZsh{}hypoth?se Conductivite joint est infinie}
          \PY{n}{Ti}\PY{o}{=}\PY{l+m+mf}{18.0}
          \PY{n}{UL}\PY{o}{=}\PY{n}{Ut}
\end{Verbatim}

    \paragraph{Calcul du rendement d'ailette
F}\label{calcul-du-rendement-dailette-f}

    

    \begin{Verbatim}[commandchars=\\\{\}]
{\color{incolor}In [{\color{incolor}229}]:} \PY{n}{m} \PY{o}{=} \PY{n}{np}\PY{o}{.}\PY{n}{sqrt}\PY{p}{(}\PY{n}{UL}\PY{o}{/}\PY{p}{(}\PY{n}{ka}\PY{o}{*}\PY{n}{deltaa}\PY{p}{)}\PY{p}{)}
\end{Verbatim}

    

    \begin{Verbatim}[commandchars=\\\{\}]
{\color{incolor}In [{\color{incolor}230}]:} \PY{n}{F} \PY{o}{=} \PY{n}{np}\PY{o}{.}\PY{n}{tanh}\PY{p}{(}\PY{p}{(}\PY{n}{m}\PY{o}{*}\PY{p}{(}\PY{n}{W}\PY{o}{\PYZhy{}}\PY{n}{D}\PY{p}{)}\PY{o}{/}\PY{l+m+mi}{2}\PY{p}{)}\PY{p}{)}\PY{o}{/}\PY{p}{(}\PY{n}{m}\PY{o}{*}\PY{p}{(}\PY{n}{W}\PY{o}{\PYZhy{}}\PY{n}{D}\PY{p}{)}\PY{o}{/}\PY{l+m+mi}{2}\PY{p}{)}
\end{Verbatim}

    \paragraph{Calcul du rendement d'absorbeur
F'}\label{calcul-du-rendement-dabsorbeur-f}

    \begin{Verbatim}[commandchars=\\\{\}]
{\color{incolor}In [{\color{incolor}231}]:} \PY{n}{Fp} \PY{o}{=} \PY{p}{(}\PY{l+m+mf}{1.0}\PY{o}{/}\PY{n}{UL}\PY{p}{)}\PY{o}{/}\PY{p}{(}\PY{n}{W}\PY{o}{*}\PY{p}{(}\PY{l+m+mf}{1.0}\PY{o}{/}\PY{p}{(}\PY{n}{UL}\PY{o}{*}\PY{p}{(}\PY{n}{D}\PY{o}{+}\PY{p}{(}\PY{n}{W}\PY{o}{\PYZhy{}}\PY{n}{D}\PY{p}{)}\PY{o}{*}\PY{n}{F}\PY{p}{)}\PY{p}{)}\PY{o}{+}\PY{n}{Rpjoint}\PY{o}{+}\PY{p}{(}\PY{l+m+mf}{1.0}\PY{o}{/}\PY{p}{(}\PY{n}{hf}\PY{o}{*}\PY{n}{const}\PY{o}{.}\PY{n}{pi}\PY{o}{*}\PY{n}{D}\PY{p}{)}\PY{p}{)}\PY{p}{)}\PY{p}{)}
          \PY{n+nb}{print}\PY{p}{(}\PY{l+s+s1}{\PYZsq{}}\PY{l+s+s1}{Fp =}\PY{l+s+s1}{\PYZsq{}}\PY{p}{,}\PY{n}{Fp}\PY{p}{)}
\end{Verbatim}

    \begin{Verbatim}[commandchars=\\\{\}]
Fp = 0.980912220305
    \end{Verbatim}

    \paragraph{Calcul du rendement d'absorbeur
F''}\label{calcul-du-rendement-dabsorbeur-f}

    \begin{Verbatim}[commandchars=\\\{\}]
{\color{incolor}In [{\color{incolor}232}]:} \PY{n}{Fpp1} \PY{o}{=} \PY{p}{(}\PY{n}{mpt}\PY{o}{*}\PY{n}{Cp}\PY{p}{)}\PY{o}{/}\PY{p}{(}\PY{n}{Ac}\PY{o}{*}\PY{n}{UL}\PY{o}{*}\PY{n}{Fp}\PY{p}{)}
          \PY{n}{Fpp2} \PY{o}{=}  \PY{l+m+mi}{1}\PY{o}{\PYZhy{}}\PY{n}{np}\PY{o}{.}\PY{n}{exp}\PY{p}{(}\PY{o}{\PYZhy{}}\PY{p}{(}\PY{n}{Ac}\PY{o}{*}\PY{n}{UL}\PY{o}{*}\PY{n}{Fp}\PY{p}{)}\PY{o}{/}\PY{p}{(}\PY{n}{mpt}\PY{o}{*}\PY{n}{Cp}\PY{p}{)}\PY{p}{)}
          \PY{n}{Fpp} \PY{o}{=} \PY{n}{Fpp1} \PY{o}{*} \PY{n}{Fpp2}
          \PY{n+nb}{print}\PY{p}{(}\PY{n}{Fpp}\PY{p}{)}
\end{Verbatim}

    \begin{Verbatim}[commandchars=\\\{\}]
0.967196774224
    \end{Verbatim}

    \paragraph{Calcul du facteur FR}\label{calcul-du-facteur-fr}

    

    \begin{Verbatim}[commandchars=\\\{\}]
{\color{incolor}In [{\color{incolor}233}]:} \PY{n}{Fr} \PY{o}{=} \PY{n}{Fpp}\PY{o}{*}\PY{n}{Fp}
          \PY{n+nb}{print}\PY{p}{(}\PY{n}{Fr}\PY{p}{)}
\end{Verbatim}

    \begin{Verbatim}[commandchars=\\\{\}]
0.948735135277
    \end{Verbatim}

    \paragraph{Calcul de la température à l'entrée du
capteur}\label{calcul-de-la-tempuxe9rature-uxe0-lentruxe9e-du-capteur}

     En isolant Tfi de la formule, on obtient,

    \begin{Verbatim}[commandchars=\\\{\}]
{\color{incolor}In [{\color{incolor}234}]:} \PY{n}{Qu} \PY{o}{=} \PY{n}{Ac}\PY{o}{*}\PY{n}{Fr}\PY{o}{*}\PY{p}{(}\PY{n}{S}\PY{o}{\PYZhy{}}\PY{n}{UL}\PY{o}{*}\PY{p}{(}\PY{n}{Ti}\PY{o}{\PYZhy{}}\PY{n}{T\PYZus{}ac}\PY{p}{)}\PY{p}{)}
          \PY{n}{Tfi} \PY{o}{=} \PY{n}{T\PYZus{}pc}\PY{o}{\PYZhy{}}\PY{p}{(}\PY{p}{(}\PY{n}{Ac}\PY{p}{)}\PY{o}{/}\PY{p}{(}\PY{n}{Fr}\PY{o}{*}\PY{n}{UL}\PY{p}{)}\PY{p}{)}\PY{o}{*}\PY{p}{(}\PY{l+m+mi}{1}\PY{o}{\PYZhy{}}\PY{n}{Fpp}\PY{p}{)}
\end{Verbatim}

    \subsubsection{Résultats}\label{ruxe9sultats}

    \begin{Verbatim}[commandchars=\\\{\}]
{\color{incolor}In [{\color{incolor}235}]:} \PY{n+nb}{print}\PY{p}{(}\PY{l+s+s1}{\PYZsq{}}\PY{l+s+s1}{a)}\PY{l+s+s1}{\PYZsq{}}\PY{p}{,}\PY{n}{Fr}\PY{p}{)}
          \PY{n+nb}{print}\PY{p}{(}\PY{l+s+s1}{\PYZsq{}}\PY{l+s+s1}{b)}\PY{l+s+s1}{\PYZsq{}}\PY{p}{,}\PY{n}{Tfi}\PY{p}{,}\PY{l+s+s1}{\PYZsq{}}\PY{l+s+s1}{°C}\PY{l+s+s1}{\PYZsq{}}\PY{p}{)}
\end{Verbatim}

    \begin{Verbatim}[commandchars=\\\{\}]
a) 0.948735135277
b) 51.9806472836 °C
    \end{Verbatim}


    % Add a bibliography block to the postdoc
    
    
    
    \end{document}
